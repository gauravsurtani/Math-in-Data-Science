% Options for packages loaded elsewhere
\PassOptionsToPackage{unicode}{hyperref}
\PassOptionsToPackage{hyphens}{url}
%
\documentclass[
]{article}
\usepackage{amsmath,amssymb}
\usepackage{iftex}
\ifPDFTeX
  \usepackage[T1]{fontenc}
  \usepackage[utf8]{inputenc}
  \usepackage{textcomp} % provide euro and other symbols
\else % if luatex or xetex
  \usepackage{unicode-math} % this also loads fontspec
  \defaultfontfeatures{Scale=MatchLowercase}
  \defaultfontfeatures[\rmfamily]{Ligatures=TeX,Scale=1}
\fi
\usepackage{lmodern}
\ifPDFTeX\else
  % xetex/luatex font selection
\fi
% Use upquote if available, for straight quotes in verbatim environments
\IfFileExists{upquote.sty}{\usepackage{upquote}}{}
\IfFileExists{microtype.sty}{% use microtype if available
  \usepackage[]{microtype}
  \UseMicrotypeSet[protrusion]{basicmath} % disable protrusion for tt fonts
}{}
\makeatletter
\@ifundefined{KOMAClassName}{% if non-KOMA class
  \IfFileExists{parskip.sty}{%
    \usepackage{parskip}
  }{% else
    \setlength{\parindent}{0pt}
    \setlength{\parskip}{6pt plus 2pt minus 1pt}}
}{% if KOMA class
  \KOMAoptions{parskip=half}}
\makeatother
\usepackage{xcolor}
\usepackage[margin=1in]{geometry}
\usepackage{color}
\usepackage{fancyvrb}
\newcommand{\VerbBar}{|}
\newcommand{\VERB}{\Verb[commandchars=\\\{\}]}
\DefineVerbatimEnvironment{Highlighting}{Verbatim}{commandchars=\\\{\}}
% Add ',fontsize=\small' for more characters per line
\usepackage{framed}
\definecolor{shadecolor}{RGB}{248,248,248}
\newenvironment{Shaded}{\begin{snugshade}}{\end{snugshade}}
\newcommand{\AlertTok}[1]{\textcolor[rgb]{0.94,0.16,0.16}{#1}}
\newcommand{\AnnotationTok}[1]{\textcolor[rgb]{0.56,0.35,0.01}{\textbf{\textit{#1}}}}
\newcommand{\AttributeTok}[1]{\textcolor[rgb]{0.13,0.29,0.53}{#1}}
\newcommand{\BaseNTok}[1]{\textcolor[rgb]{0.00,0.00,0.81}{#1}}
\newcommand{\BuiltInTok}[1]{#1}
\newcommand{\CharTok}[1]{\textcolor[rgb]{0.31,0.60,0.02}{#1}}
\newcommand{\CommentTok}[1]{\textcolor[rgb]{0.56,0.35,0.01}{\textit{#1}}}
\newcommand{\CommentVarTok}[1]{\textcolor[rgb]{0.56,0.35,0.01}{\textbf{\textit{#1}}}}
\newcommand{\ConstantTok}[1]{\textcolor[rgb]{0.56,0.35,0.01}{#1}}
\newcommand{\ControlFlowTok}[1]{\textcolor[rgb]{0.13,0.29,0.53}{\textbf{#1}}}
\newcommand{\DataTypeTok}[1]{\textcolor[rgb]{0.13,0.29,0.53}{#1}}
\newcommand{\DecValTok}[1]{\textcolor[rgb]{0.00,0.00,0.81}{#1}}
\newcommand{\DocumentationTok}[1]{\textcolor[rgb]{0.56,0.35,0.01}{\textbf{\textit{#1}}}}
\newcommand{\ErrorTok}[1]{\textcolor[rgb]{0.64,0.00,0.00}{\textbf{#1}}}
\newcommand{\ExtensionTok}[1]{#1}
\newcommand{\FloatTok}[1]{\textcolor[rgb]{0.00,0.00,0.81}{#1}}
\newcommand{\FunctionTok}[1]{\textcolor[rgb]{0.13,0.29,0.53}{\textbf{#1}}}
\newcommand{\ImportTok}[1]{#1}
\newcommand{\InformationTok}[1]{\textcolor[rgb]{0.56,0.35,0.01}{\textbf{\textit{#1}}}}
\newcommand{\KeywordTok}[1]{\textcolor[rgb]{0.13,0.29,0.53}{\textbf{#1}}}
\newcommand{\NormalTok}[1]{#1}
\newcommand{\OperatorTok}[1]{\textcolor[rgb]{0.81,0.36,0.00}{\textbf{#1}}}
\newcommand{\OtherTok}[1]{\textcolor[rgb]{0.56,0.35,0.01}{#1}}
\newcommand{\PreprocessorTok}[1]{\textcolor[rgb]{0.56,0.35,0.01}{\textit{#1}}}
\newcommand{\RegionMarkerTok}[1]{#1}
\newcommand{\SpecialCharTok}[1]{\textcolor[rgb]{0.81,0.36,0.00}{\textbf{#1}}}
\newcommand{\SpecialStringTok}[1]{\textcolor[rgb]{0.31,0.60,0.02}{#1}}
\newcommand{\StringTok}[1]{\textcolor[rgb]{0.31,0.60,0.02}{#1}}
\newcommand{\VariableTok}[1]{\textcolor[rgb]{0.00,0.00,0.00}{#1}}
\newcommand{\VerbatimStringTok}[1]{\textcolor[rgb]{0.31,0.60,0.02}{#1}}
\newcommand{\WarningTok}[1]{\textcolor[rgb]{0.56,0.35,0.01}{\textbf{\textit{#1}}}}
\usepackage{graphicx}
\makeatletter
\def\maxwidth{\ifdim\Gin@nat@width>\linewidth\linewidth\else\Gin@nat@width\fi}
\def\maxheight{\ifdim\Gin@nat@height>\textheight\textheight\else\Gin@nat@height\fi}
\makeatother
% Scale images if necessary, so that they will not overflow the page
% margins by default, and it is still possible to overwrite the defaults
% using explicit options in \includegraphics[width, height, ...]{}
\setkeys{Gin}{width=\maxwidth,height=\maxheight,keepaspectratio}
% Set default figure placement to htbp
\makeatletter
\def\fps@figure{htbp}
\makeatother
\setlength{\emergencystretch}{3em} % prevent overfull lines
\providecommand{\tightlist}{%
  \setlength{\itemsep}{0pt}\setlength{\parskip}{0pt}}
\setcounter{secnumdepth}{-\maxdimen} % remove section numbering
\ifLuaTeX
  \usepackage{selnolig}  % disable illegal ligatures
\fi
\IfFileExists{bookmark.sty}{\usepackage{bookmark}}{\usepackage{hyperref}}
\IfFileExists{xurl.sty}{\usepackage{xurl}}{} % add URL line breaks if available
\urlstyle{same}
\hypersetup{
  pdftitle={Spotify-data-EDA},
  pdfauthor={Gaurav Surtani},
  hidelinks,
  pdfcreator={LaTeX via pandoc}}

\title{Spotify-data-EDA}
\author{Gaurav Surtani}
\date{2023-11-04}

\begin{document}
\maketitle

\hypertarget{read-data-from-csv-file}{%
\subsubsection{Read data from CSV file:}\label{read-data-from-csv-file}}

\begin{Shaded}
\begin{Highlighting}[]
\NormalTok{data }\OtherTok{\textless{}{-}} \FunctionTok{read.csv}\NormalTok{(}\StringTok{"spotify{-}2023.csv"}\NormalTok{)}
\end{Highlighting}
\end{Shaded}

\hypertarget{know-the-dimensions}{%
\subsubsection{Know the dimensions:}\label{know-the-dimensions}}

\begin{Shaded}
\begin{Highlighting}[]
\FunctionTok{dim}\NormalTok{(data)}
\end{Highlighting}
\end{Shaded}

\begin{verbatim}
## [1] 953  24
\end{verbatim}

\hypertarget{check-the-new-dimensions-of-the-cleaned-data}{%
\subsubsection{Check the new dimensions of the cleaned
data:}\label{check-the-new-dimensions-of-the-cleaned-data}}

\begin{Shaded}
\begin{Highlighting}[]
\NormalTok{spotify\_data }\OtherTok{\textless{}{-}} \FunctionTok{na.omit}\NormalTok{(data)}
\FunctionTok{dim}\NormalTok{(spotify\_data)}
\end{Highlighting}
\end{Shaded}

\begin{verbatim}
## [1] 953  24
\end{verbatim}

\hypertarget{exploring-spotify-music-trends}{%
\section{Exploring Spotify Music
Trends}\label{exploring-spotify-music-trends}}

\hypertarget{introduction}{%
\subsection{Introduction}\label{introduction}}

I'm interested in understanding what musical attributes and trends are
associated with popularity and success on Spotify. Specifically, I want
to explore how release date, tempo, danceability, energy, etc. relate to
metrics like streams and playlist additions. This could shed light on
what current listeners value in music.\\
I have chosen the Spotify dataset as it provides a comprehensive list of
the most famous songs of 2023. I am curious to explore various aspects
of popular music and how songs perform across different streaming
platforms.

\hypertarget{data}{%
\subsection{Data}\label{data}}

\hypertarget{summary}{%
\subsubsection{Summary}\label{summary}}

The data comes from a CSV file containing details on songs released in
2022-2023 that appeared on Spotify charts and playlists. It has 950+
rows and 23 columns, with each row representing a song.

\textbf{Key variables i want to focus on in this dataset is as belows}:
- release\_date: date song was released - streams: total streams on
Spotify - playlist\_adds: number of Spotify playlists song was added to
- bpm: beats per minute - danceability: Spotify danceability score -
energy: Spotify energy score - key: song key - mode: major/minor

The data was web scraped and compiled in January 2023.

\begin{itemize}
\item
  \textbf{Data Source}: The dataset is sourced from Kaggle
  \href{https://www.kaggle.com/datasets/nelgiriyewithana/top-spotify-songs-2023/data}{Link}
  and Spotify and contains information about popular songs in 2023.
\item
  \textbf{Data Collection}: The data was collected through a combination
  of sources, including Spotify's internal databases, music charts, and
  streaming statistics, through web scraping or API calls to gather
  additional information done in Kaggle
\item
  \textbf{Cases}: Each row in the dataset represents a unique song. It
  provides detailed information about each song's attributes,
  popularity, and presence on various music platforms.
\item
  \textbf{Variables}: (Referenced from Kaggle variable list)

  \begin{itemize}
  \tightlist
  \item
    \texttt{track\_name}: Name of the song.
  \item
    \texttt{artist(s)\_name}: Name of the artist(s) of the song.
  \item
    \texttt{artist\_count}: Number of artists contributing to the song.
  \item
    \texttt{released\_year}: Year when the song was released.
  \item
    \texttt{released\_month}: Month when the song was released.
  \item
    \texttt{released\_day}: Day of the month when the song was released.
  \item
    \texttt{in\_spotify\_playlists}: Number of Spotify playlists the
    song is included in.
  \item
    \texttt{in\_spotify\_charts}: Presence and rank of the song on
    Spotify charts.
  \item
    \texttt{streams}: Total number of streams on Spotify.
  \item
    \texttt{in\_apple\_playlists}: Number of Apple Music playlists the
    song is included in.
  \item
    \texttt{in\_apple\_charts}: Presence and rank of the song on Apple
    Music charts.
  \item
    \texttt{in\_deezer\_playlists}: Number of Deezer playlists the song
    is included in.
  \item
    \texttt{in\_deezer\_charts}: Presence and rank of the song on Deezer
    charts.
  \item
    \texttt{in\_shazam\_charts}: Presence and rank of the song on Shazam
    charts.
  \item
    \texttt{bpm}: Beats per minute, a measure of song tempo.
  \item
    \texttt{key}: Key of the song.
  \item
    \texttt{mode}: Mode of the song (major or minor).
  \item
    \texttt{danceability\_\%}: Percentage indicating how suitable the
    song is for dancing.
  \item
    \texttt{valence\_\%}: Positivity of the song's musical content.
  \item
    \texttt{energy\_\%}: Perceived energy level of the song.
  \item
    \texttt{acousticness\_\%}: Amount of acoustic sound in the song.
  \item
    \texttt{instrumentalness\_\%}: Amount of instrumental content in the
    song.
  \item
    \texttt{liveness\_\%}: Presence of live performance elements.
  \item
    \texttt{speechiness\_\%}: Amount of spoken words in the song.
  \end{itemize}
\item
  \textbf{Type of Study}: This dataset is observational, as it provides
  information about songs and their attributes without any controlled
  experiments.
\end{itemize}

\textbf{--------------------------------------------------------------------------------------}

\hypertarget{exploratory-data-analysis}{%
\section{Exploratory Data Analysis:}\label{exploratory-data-analysis}}

\hypertarget{visualizations}{%
\subsection{Visualizations:}\label{visualizations}}

\hypertarget{histograms}{%
\subsection{Histograms:}\label{histograms}}

I chose to start with Histograms of Streams and Playlist Additions
because:\\
- They provide an overview of the data distribution. You can immediately
see if the data is skewed, has a normal distribution, or has multiple
modes.\\
- They help identify outliers.If there's a long tail, that could suggest
the presence of outliers. We may have to remove some outliers at the end
because they might possibly skew the results towards them.\\
- They are a precursor to data cleaning. If you spot any anomalies, such
as unexpected spikes that don't correspond to the real-world behavior of
the data, this might indicate errors or noise in the data collection
process that need to be addressed.

\hypertarget{histograms-for-streams-in-millions-and-playlist-adds}{%
\subsubsection{1. Histograms for Streams in millions and Playlist
Adds:}\label{histograms-for-streams-in-millions-and-playlist-adds}}

\textbf{We scale down to the stream to in millions,convert to int and
clean-up NA's to improve data clarity}

\begin{Shaded}
\begin{Highlighting}[]
\NormalTok{spotify\_data}\SpecialCharTok{$}\NormalTok{streams }\OtherTok{\textless{}{-}} \FunctionTok{as.numeric}\NormalTok{(}\FunctionTok{as.character}\NormalTok{(spotify\_data}\SpecialCharTok{$}\NormalTok{streams))}
\end{Highlighting}
\end{Shaded}

\begin{verbatim}
## Warning: NAs introduced by coercion
\end{verbatim}

\begin{Shaded}
\begin{Highlighting}[]
\NormalTok{spotify\_data }\OtherTok{\textless{}{-}}\NormalTok{ spotify\_data[}\SpecialCharTok{!}\FunctionTok{is.na}\NormalTok{(spotify\_data}\SpecialCharTok{$}\NormalTok{streams), ]}
\NormalTok{spotify\_data}\SpecialCharTok{$}\NormalTok{streams\_in\_millions }\OtherTok{\textless{}{-}}\NormalTok{ spotify\_data}\SpecialCharTok{$}\NormalTok{streams}\SpecialCharTok{/}\DecValTok{1000000}
\FunctionTok{summary}\NormalTok{(spotify\_data}\SpecialCharTok{$}\NormalTok{streams\_in\_millions)}
\end{Highlighting}
\end{Shaded}

\begin{verbatim}
##     Min.  1st Qu.   Median     Mean  3rd Qu.     Max. 
##    0.003  141.636  290.531  514.137  673.869 3703.895
\end{verbatim}

\begin{Shaded}
\begin{Highlighting}[]
\FunctionTok{ggplot}\NormalTok{(spotify\_data, }\FunctionTok{aes}\NormalTok{(}\AttributeTok{x =}\NormalTok{ streams\_in\_millions)) }\SpecialCharTok{+}
  \FunctionTok{geom\_histogram}\NormalTok{(}\AttributeTok{binwidth =} \DecValTok{100}\NormalTok{, }\AttributeTok{fill =} \StringTok{"blue"}\NormalTok{, }\AttributeTok{color =} \StringTok{"gray"}\NormalTok{) }\SpecialCharTok{+}
  \FunctionTok{labs}\NormalTok{(}\AttributeTok{title =} \StringTok{"Distribution of Streams"}\NormalTok{)}
\end{Highlighting}
\end{Shaded}

\includegraphics{spotify-data-markdown-EDA_files/figure-latex/unnamed-chunk-4-1.pdf}

\begin{Shaded}
\begin{Highlighting}[]
\FunctionTok{ggplot}\NormalTok{(spotify\_data, }\FunctionTok{aes}\NormalTok{(}\AttributeTok{x =}\NormalTok{ in\_spotify\_playlists)) }\SpecialCharTok{+}
  \FunctionTok{geom\_histogram}\NormalTok{(}\AttributeTok{binwidth =} \DecValTok{1000}\NormalTok{, }\AttributeTok{fill =} \StringTok{"green"}\NormalTok{, }\AttributeTok{color =} \StringTok{"black"}\NormalTok{) }\SpecialCharTok{+}
  \FunctionTok{labs}\NormalTok{(}\AttributeTok{title =} \StringTok{"Distribution of Songs added in playlists"}\NormalTok{)}
\end{Highlighting}
\end{Shaded}

\includegraphics{spotify-data-markdown-EDA_files/figure-latex/unnamed-chunk-4-2.pdf}

\begin{enumerate}
\def\labelenumi{\arabic{enumi}.}
\tightlist
\item
  \textbf{Distribution of Streams}:

  \begin{itemize}
  \tightlist
  \item
    The histogram for \texttt{Streams} is \textbf{right-skewed}. Most
    songs have a relatively small number of streams (in millions), while
    a few songs have a very high number of streams.
  \item
    There is a noticeable peak in the distribution, which again suggests
    that a higher number of songs have fewer streams.
  \item
    The long tail to the right indicates that while most songs don't
    achieve extremely high stream counts, there are a select few that
    do.
  \end{itemize}
\item
  \textbf{Distribution of Playlist Adds}:

  \begin{itemize}
  \tightlist
  \item
    The histogram for \texttt{Songs\ added\ in\ playlists} shows a
    \textbf{heavily right-skewed} distribution. This indicates that a
    large number of songs have a relatively small number of playlist
    additions, while only a few songs have a very high number of
    additions.
  \item
    The peak at the left suggests that the most common number of
    playlist additions is low, near zero.
  \item
    There are a few outliers with a very high number of playlist adds,
    but these are exceptional.
  \item
    The distribution suggests that it is relatively rare for songs to be
    added to a large number of playlists.
  \end{itemize}
\end{enumerate}

\hypertarget{scatterplots}{%
\subsection{ScatterPlots:}\label{scatterplots}}

\hypertarget{song-added-to-playlist-vs.-energy-level-of-the-song}{%
\subsubsection{1. Song Added to Playlist vs.~Energy Level of the
song:}\label{song-added-to-playlist-vs.-energy-level-of-the-song}}

\begin{Shaded}
\begin{Highlighting}[]
\NormalTok{spotify\_data\_lessthan10000\_playlist }\OtherTok{\textless{}{-}}\NormalTok{ spotify\_data }\SpecialCharTok{\%\textgreater{}\%} 
  \FunctionTok{filter}\NormalTok{(in\_spotify\_playlists }\SpecialCharTok{\textless{}=} \DecValTok{10000}\NormalTok{)}

\FunctionTok{ggplot}\NormalTok{(spotify\_data\_lessthan10000\_playlist, }\FunctionTok{aes}\NormalTok{(}\AttributeTok{x =}\NormalTok{ energy\_., }\AttributeTok{y =}\NormalTok{ in\_spotify\_playlists)) }\SpecialCharTok{+}
  \FunctionTok{geom\_point}\NormalTok{(}\AttributeTok{alpha =} \FloatTok{0.75}\NormalTok{, }\AttributeTok{color =} \StringTok{"blue"}\NormalTok{) }\SpecialCharTok{+}
  \FunctionTok{labs}\NormalTok{(}\AttributeTok{title =} \StringTok{"Song Added to Playlist  vs. Energy Level of the song"}\NormalTok{,}
       \AttributeTok{x =} \StringTok{"Energy Level"}\NormalTok{,}
       \AttributeTok{y =} \StringTok{"Playlists that song is present in"}\NormalTok{)}\SpecialCharTok{+}
  \FunctionTok{xlim}\NormalTok{(}\DecValTok{0}\NormalTok{,}\DecValTok{100}\NormalTok{)}
\end{Highlighting}
\end{Shaded}

\includegraphics{spotify-data-markdown-EDA_files/figure-latex/unnamed-chunk-5-1.pdf}

\textbf{Song Added to Playlist vs.~Energy Level of the song}: - The plot
shows a wide spread of points, suggesting \emph{there isn't a clear
linear relationship} between the energy level of songs and the number of
playlists they appear in. - While songs of all energy levels appear to
have a chance of being added to a range of playlists, there is a
concentration of songs with lower playlist presence, indicating that
most songs, regardless of energy, tend to have a lower number of
playlist adds. - There are some songs with high energy levels that also
have a higher number of playlist adds, but these are not the majority,
indicating that high energy alone does not guarantee a higher presence
in playlists.

\hypertarget{danceability-of-the-song-vs.-tempo}{%
\subsubsection{2. Danceability of the Song
vs.~Tempo:}\label{danceability-of-the-song-vs.-tempo}}

\begin{Shaded}
\begin{Highlighting}[]
\FunctionTok{ggplot}\NormalTok{(spotify\_data, }\FunctionTok{aes}\NormalTok{(}\AttributeTok{x =}\NormalTok{ danceability\_., }\AttributeTok{y =}\NormalTok{ bpm )) }\SpecialCharTok{+}
  \FunctionTok{geom\_point}\NormalTok{(}\AttributeTok{alpha =} \FloatTok{0.6}\NormalTok{, }\AttributeTok{color =} \StringTok{"purple"}\NormalTok{) }\SpecialCharTok{+}
  \FunctionTok{labs}\NormalTok{(}\AttributeTok{title =} \StringTok{"Danceability of the Song vs. Tempo"}\NormalTok{,}
       \AttributeTok{x =} \StringTok{"Danceability of the Song"}\NormalTok{,}
       \AttributeTok{y =} \StringTok{"Tempo (BPM)"}\NormalTok{)}\SpecialCharTok{+}
  \FunctionTok{xlim}\NormalTok{(}\DecValTok{0}\NormalTok{, }\DecValTok{100}\NormalTok{) }
\end{Highlighting}
\end{Shaded}

\includegraphics{spotify-data-markdown-EDA_files/figure-latex/unnamed-chunk-6-1.pdf}

\textbf{Danceability of the Song vs.~Tempo}: - This plot shows a broad
and relatively uniform distribution of points across the range of
danceability scores between 25 and 80. - There is danceability on bpm in
range of 80 - 180. The spread of BPM across danceability scores suggests
that songs with a wide range of tempos can be danceable, and high
danceability is not confined to a narrow tempo range. - There doesn't
seem to be a strong correlation between danceability and tempo based on
this plot, indicating that tempo might not be a defining factor in
danceability, or at least that danceable songs can come at a variety of
tempos.

\hypertarget{scatterplots-1}{%
\subsection{ScatterPlots:}\label{scatterplots-1}}

\hypertarget{average-tempo-by-musical-key}{%
\subsubsection{1. Average Tempo by Musical
Key:}\label{average-tempo-by-musical-key}}

Which songs have highest tempo based on the musical key. We answer these
type of questions using this plot. By using this plot, you are able to
present a clear and statistically grounded picture of how tempo varies
by musical key in the songs from your Spotify data. It's a more nuanced
view than simply plotting the raw data or the means without confidence
intervals, as it takes into account the precision of your estimates.

\begin{Shaded}
\begin{Highlighting}[]
\NormalTok{spotify\_data\_filtered\_emptyKey }\OtherTok{\textless{}{-}}\NormalTok{ spotify\_data }\SpecialCharTok{\%\textgreater{}\%}
  \FunctionTok{filter}\NormalTok{(}\SpecialCharTok{!}\FunctionTok{is.na}\NormalTok{(key) }\SpecialCharTok{\&} \SpecialCharTok{!}\FunctionTok{is.na}\NormalTok{(bpm) }\SpecialCharTok{\&}\NormalTok{ bpm }\SpecialCharTok{!=} \DecValTok{0} \SpecialCharTok{\&}\NormalTok{ key}\SpecialCharTok{!=}\StringTok{\textquotesingle{}\textquotesingle{}}\NormalTok{)}

\FunctionTok{ggplot}\NormalTok{(spotify\_data\_filtered\_emptyKey, }\FunctionTok{aes}\NormalTok{(}\AttributeTok{x =} \FunctionTok{as.factor}\NormalTok{(key), }\AttributeTok{y =}\NormalTok{ bpm)) }\SpecialCharTok{+}
  \FunctionTok{geom\_pointrange}\NormalTok{(}\AttributeTok{stat =} \StringTok{"summary"}\NormalTok{, }\AttributeTok{fun.data =} \StringTok{"mean\_cl\_normal"}\NormalTok{, }\AttributeTok{color =} \StringTok{"blue"}\NormalTok{) }\SpecialCharTok{+}
  \FunctionTok{labs}\NormalTok{(}\AttributeTok{title =} \StringTok{"Average Tempo by Musical Key"}\NormalTok{,}
       \AttributeTok{x =} \StringTok{"Musical Key"}\NormalTok{,}
       \AttributeTok{y =} \StringTok{"Tempo (BPM)"}\NormalTok{)}
\end{Highlighting}
\end{Shaded}

\includegraphics{spotify-data-markdown-EDA_files/figure-latex/unnamed-chunk-7-1.pdf}

The graph suggests a stable trend in song tempos across musical keys,
with average BPMs closely grouped between 120 and 130. Variability
within each key is moderate, and no particular key is associated with a
distinctly faster or slower tempo. This indicates that a song's key is
likely not a major factor in determining its tempo.

The \texttt{geom\_pointrange} plot you have created is effectively a way
to visualize the mean tempo (\texttt{bpm}) for songs in each musical key
(\texttt{key}) along with the confidence intervals for those means.

\hypertarget{what-questions-this-plot-answers}{%
\subsubsection{What Questions This Plot
Answers:}\label{what-questions-this-plot-answers}}

\begin{itemize}
\tightlist
\item
  \textbf{What is the average tempo for songs in each musical key?} You
  can compare the central tendency (mean bpm) across different keys.
\item
  \textbf{How much variability is there in the tempo of songs within
  each key?} The length of the vertical lines (the point ranges)
  indicates the confidence interval for the mean, which reflects
  variability. A longer line means more variability; a shorter line
  means less.
\item
  \textbf{Are there significant differences in tempo between keys?} If
  the confidence intervals for two keys don't overlap, it suggests a
  significant difference in the average tempos between those keys.
\item
  \textbf{Are certain keys associated with faster or slower tempos?}
  This can be seen by the position of the point on the y-axis (tempo).
\end{itemize}

\hypertarget{conclusion-i-draw-from-the-given-analysis}{%
\subsection{Conclusion I draw from the given
analysis:}\label{conclusion-i-draw-from-the-given-analysis}}

I calculated summary statistics and visualized distributions of key
variables:

\begin{itemize}
\item
  Most songs are relatively recent, released in 2022 or 2023. Streams
  and playlist adds are right skewed, with most songs having
  \textless500M streams and \textless200 playlist adds.
\item
  There is danceability on bpm in range of 80 - 180. The spread of BPM
  across danceability scores suggests that songs with a wide range of
  tempos can be danceable, and high danceability is not confined to a
  narrow tempo range.
\item
  Songs released more recently tend to have fewer streams, likely
  because they've had less time to accumulate them. Songs with more
  playlist adds also tend to have more streams.
\item
  While songs of all energy levels appear to have a chance of being
  added to a range of playlists, there is a concentration of songs with
  lower playlist presence, indicating that most songs, regardless of
  energy, tend to have a lower number of playlist adds.
\end{itemize}

\textbf{--------------------------------------------------------------------------------------}

\hypertarget{future-questions-that-i-can-ask-from-the-complete-dataset}{%
\subsection{Future Questions that I can ask from the complete
dataset!}\label{future-questions-that-i-can-ask-from-the-complete-dataset}}

I'd like to test hypotheses about how song attributes relate to
popularity:

\begin{enumerate}
\def\labelenumi{\arabic{enumi}.}
\item
  Are songs with higher danceability scores streamed more?
\item
  Do songs with higher energy have more playlist adds?
\item
  Are songs in a major key streamed more than songs in a minor key?
\end{enumerate}

\textbf{--------------------------------------------------------------------------------------}

\end{document}
